% http://www.ctan.org/tex-archive/macros/latex/contrib/beamer/examples
% http://latex.artikel-namsu.de/english/beamer-examples.html

\documentclass{beamer}
\usepackage{amsmath}
\usepackage{amssymb}
\usepackage{bm}
\usepackage{fancybox, graphicx}
\usepackage{listings}
\usepackage{tikz} % Diagrams
\usepackage{color}
\usepackage{xcolor}
\usepackage{textcomp} % See https://tex.stackexchange.com/questions/145416/how-to-have-straight-single-quotes-in-lstlistings

\lstset{language=Python,upquote=true} % Format listings as appropriate for Python. Inexplicably we get problems if the language is set as part of the \begin{lstlisting} command.

% https://tex.stackexchange.com/questions/36030/how-to-make-a-single-word-look-as-some-code
\definecolor{light-gray}{gray}{0.95}
%\definecolor{light-green}{RGB}{0,250,0}
\definecolor{light-green}{cmyk}{0.07,0,0.07,0}
\newcommand{\command}[1]{\colorbox{light-gray}{\texttt{#1}}}
\newcommand{\filename}[1]{\colorbox{light-green}{\texttt{#1}}}


\usetheme{boxes}
\usecolortheme{beaver}


\title{Introduction to the UNIX Command Line}

\author{Lorne~Whiteway \\ lorne.whiteway.13@ucl.ac.uk}

\institute[UCL]
{
  Astrophysics Group\\
  Department of Physics and Astronomy\\
  University College London
}
\date
{30 January 2025}


\begin{document}

\frame{\titlepage}

\section{Introduction}

\begin{frame}{Where to find this presentation}
    Find the presentation at \alert{\url{https://tinyurl.com/ytt3kdm3}}.\\
\end{frame}


\begin{frame}{Overall goals of presentation}
  \begin{itemize}
    \item How to access UCL UNIX systems
    \item How to use the UNIX command line
    \item Pointers to where to get more information (courses, web, etc.)
  \end{itemize}
\end{frame}


\begin{frame}{Information on the Web}
  \begin{block}{Astrophysics Wiki}
    \url{https://liveuclac.sharepoint.com/sites/PhysAstAstPhysGrp}
    This Wiki is freely viewable and editable by all members of the department. Please use it to record information that you think will be useful to others (including your future self). Be bold!
  \end{block}


  \begin{block}{UCL Research Computing Platforms}
    \url{https://www.rc.ucl.ac.uk/}
  \end{block}

  
  \begin{block}{Stack Overflow}
    \url{http://stackoverflow.com/}  \ (But will it survive ChatGPT?)
  \end{block}
  
\end{frame}

\begin{frame}{Computing Environment for Astrophysics}
  \begin{itemize}
  \item Large datasets requiring substantial processing followed by sophisticated statistical analysis
  \item Calculations often done on specialised `high-performance computing' (HPC) machines having large filesystems and large RAM; calculations are often broken into pieces that can be run simultaneously (`in parallel') across many processors.
  \item Much useful software is made freely available within the community. Software quality is usually high; documentation quality is more variable.
  \item Many users write their own software.
  \end{itemize}
\end{frame}

\begin{frame}{Local Computing Environment}
  \begin{block}{You will have your own local machine, which might be:}
    \begin{itemize}
      \item PC (Windows)
      \item Mac
      \item Linux
    \end{itemize}
  \end{block}
  \begin{block}{Also there are shared Linux machines:}
    \begin{itemize}
      \item General purpose Astrophysics servers available from outside UCL: \alert{\texttt{zuserver1}} and \alert{\texttt{zuserver2}}
      \item UCL Cosmology HPC clusters: \alert{\texttt{splinter}} and \alert{\texttt{hypatia}}
      \item Other UCL clusters: \alert{\texttt{Myriad}} and \alert{\texttt{Kathleen}}
      \item National clusters: \alert{\texttt{DiRAC}} (UK) and \alert{\texttt{NERSC}} (US)
    \end{itemize}
  \end{block}
\end{frame}

\begin{frame}{Work patterns}
  \begin{block}{Several work patterns are possible:}
    \begin{itemize}
      \item Write and test a program on your local machine; use the local machine to remotely connect to a server; upload the program to the server and run it there;
      \item Or do all your work locally (requires small data sets);
      \item Or use the local machine to remotely connect to a server and do all your work there.
    \end{itemize}
  \end{block}
\end{frame}

\begin{frame}{Accessing remote machines}
  \begin{block}{Credentials}
    \begin{itemize}
      \item You will need a \textit{username} and \textit{password} for any machine that you want to access.
      \item Contact Edd Edmondson (e.edmondson@ucl.ac.uk) or John Deacon (j.deacon@ucl.ac.uk) to get these credentials.
    \end{itemize}
  \end{block}
  \begin{block}{The full names of the Astro servers are:}
    \begin{itemize}
      \item \texttt{zuserver1.star.ucl.ac.uk}
      \item \texttt{zuserver2.star.ucl.ac.uk}
      \item \texttt{splinter-login.star.ucl.ac.uk}
      \item \texttt{hypatia-login.hpc.phys.ucl.ac.uk}
    \end{itemize}
  \end{block}
\end{frame}


\begin{frame}{Software for connecting}
  \begin{block}{How to connect to a shared machine}
    \begin{itemize}
      \item Windows PC: use \command{PuTTY} (or \command{MobaXterm}, which uses \command{PuTTY});
      \item Mac: go to the Terminal window and use \command{ssh};
      \item Linux machine: go to the Terminal window and use \command{ssh}.
    \end{itemize}
  \end{block}
\end{frame}

\begin{frame}{Visibility}
  \begin{block}{}
    \begin{itemize}
    \item \texttt{zuserver1} and \texttt{zuserver2} can be seen from anywhere.
    \item If you are on the UCL network then you can see any Astro or UCL HPC server.
    \item If you are not on the UCL network then to connect to Astro or UCL HPC servers (except \texttt{zuserver1} and \texttt{zuserver2}) you must set up a \textit{Virtual Private Environment}. For details see \url{https://www.ucl.ac.uk/isd/services/get-connected/ucl-virtual-private-network-vpn}. An alternative is to logon to \texttt{zuserver1} and from there logon to the server.
    \end{itemize}
  \end{block}
\end{frame}



%\begin{frame}{Two methods for accessing splinter}
%  \begin{figure}
%    \begin{center}
%      \begin{tikzpicture}[font=\ttfamily]
%        \node at (-2.5, 6) {Option A:};
%        \node at (2.5, 6) {Option B:};
%        \node (LM1) at (-2.5,5) [rectangle,draw] {local machine};
%        \node (Z) at (-2.5,3) [rectangle,draw] {zuserver1};
%        \node (S1) at (-2.5,1) [rectangle,draw] {splinter};
%        \node (LM2) at (2.5,5) [rectangle,draw,align=center] {local machine \\ on UCL network};
%        \node (S2) at (2.5,1) [rectangle,draw] {splinter};
%        \node at (0,3) {or};
%        \draw [->,thick] (LM1.south) -- (Z.north) node [midway,left,align=center] {PuTTY \\ or ssh};
%        \draw [->,thick] (Z.south) -- (S1.north) node [midway,left] {ssh};
%        \draw [->,thick] (LM2.south) -- (S2.north) node [midway,right,align=center] {PuTTY \\ or ssh};
%      \end{tikzpicture}
%    \end{center}
%  \end{figure}
%\end{frame}



\begin{frame}{Using PuTTY for remote connections from Windows}
  \begin{itemize}
    \item If you don't have PuTTY you can download it from \url{http://www.putty.org/}.
    \item On the `Connection/SSH/X11' tab, click on `enable X11 forwarding' and set `X display location' to `localhost:0' - this is necessary for handling graphical output.
    \item On the Session tab, set the Host Name as appropriate e.g. \texttt{zuserver1.star.ucl.ac.uk}.
  \end{itemize}
\end{frame}

\begin{frame}{Using ssh for remote connections from Mac and Linux}
  \begin{itemize}
    \item Syntax: \command{ssh -YC username@servername}
    \item The `Y' option is necessary for handling graphical output.
  \end{itemize}
\end{frame}

\begin{frame}{X-Windows client}
  \begin{itemize}
    \item If the remote program that you are running produces graphical output, then you must have a program (an `X-Windows client') running on your local machine to display this graphical output.
    \item On Windows you can use XMing (\url{https://sourceforge.net/projects/xming/}) or Exceed (available on the UCL Desktop).
    \item On Mac you can use XQuartz.
    \item On Linux you don't need to do anything special - the graphical interface is already an X-server.
  \end{itemize}
\end{frame}


\section{Linux}

\begin{frame}{Linux: Command shell}
  \begin{itemize}
    \item{In Linux you will use a `command shell'.}
    \item{This is a text-based environment in which you type commands and receive text output.}
    \item{Not GUI! Reflects the hardware limitations current when Unix was created. Low-tech and reliable e.g. for remote access.}
    \item{Various command shell programs are used: \command{bash}, \command{csh}, \command{tcsh}, \command{zsh}, etc. To see which one you are using, call \command{echo \$0}.}
    \item{This presentation assumes \command{bash}! Other shells may use different names for some of the commands discussed.}
   \end{itemize}
\end{frame}

\begin{frame}{Linux: Directory structure}
  \begin{itemize}
    \item{Everything is organised around files (which may be data files or program files i.e. instructions to be executed).}
    \item{Files live in directories. There is a hierarchical tree structure of directories.}
    \item{The \textit{root} directory (the base of the directory tree) is called \filename{/}}.
    \item{Sample file name: \filename{/home/ucapwhi/foo.txt}}
    \item{Note use of slash `\texttt{/}', not backslash `\texttt{\symbol{92}}' as in Windows.}
    \item{Case sensitivity: `Foo' and `foo' are different strings.}
  \end{itemize}
\end{frame}

\begin{frame}{Linux: Working directory}
  \begin{itemize}
    \item{The shell is always pointed at one particular directory, known as the \textit{working directory}.}
    \item{Use \command{pwd} (`print working directory') to see the current working directory.}
    \item{Refer to files in the working directory simply via the file name (example \filename{foo.txt}) and refer to files in other directories by directory name plus file name (example \filename{/home/ucapwhi/foo.txt}).}
  \end{itemize}
\end{frame}


\begin{frame}{Linux: Abbreviations for directories}
  \begin{itemize}
    \item{Full stop \filename{.} is an abbreviation for the working directory.}
    \item{Two full stops \filename{..} is an abbreviation for the parent of the working directory.}
    \item{Hyphen \filename{-} is an abbreviation of the previous working directory -- useful if you need to flip back and forth between directories!}
    \item{Tilde \filename{\textasciitilde} is an abbreviation of the user's \textit{home} directory e.g. \filename{/home/ucapwhi/}. Many configuration files are located here by default.}
  \end{itemize}
\end{frame}


\begin{frame}{Linux: Navigating the directory tree}
  \begin{itemize}
    \item{Use \command{cd} to change working directory. Example: \command{cd ../data/des/}.}
    \item{Use \command{ls} to list the files in the current working directory; \\use \command{ls <dir>} to list the files in another directory.}
  \end{itemize}
\end{frame}


\begin{frame}{Linux: Keyboard shortcuts}
  \begin{itemize}
    \item{Linux has several shortcuts that make it efficient to use the keyboard to type commands.}
    \item{Keyboard interfaces predate more modern graphical interfaces; they are less friendly for new users, but are low-tech, robust, and very efficient for experienced users.}
  \end{itemize}
\end{frame}

\begin{frame}{Linux: Keyboard shortcuts: tab}
  \begin{itemize}
    \item{Use the \command{tab} key to autocomplete commands, directory names and filenames.}
    \item{If what you have typed so far doesn't have a unique autocompletion, then it will complete up to the first ambiguous character.}
    \item{This will influence your naming conventions for directories and files!}
  \end{itemize}
\end{frame}

\begin{frame}{Linux: Keyboard shortcuts: up and down arrows}
  \begin{itemize}
    \item{Use \command{up arrow} to scroll backwards through previous commands, and then \command{down arrow} to scroll forwards again.}
    \item{Follow with \command{Enter} to execute an old command that you have scrolled back to, or \command{ctrl+c} to cancel.}
  \end{itemize}
\end{frame}

\begin{frame}{Linux: Keyboard shortcuts: ctrl+r}
  \begin{itemize}
    \item{Use \command{ctrl+r} to search backwards through previous commands (\textit{reverse-i-search}).}
    \item{Type a substring (not necessarily initial) of the sought-for command.}
    \item{Example: \command{ctrl+r push} will bring up the most recent command that included the substring \command{push}.}
    \item{Follow with \command{Enter} to execute, \command{ctrl+c} to cancel, or \command{ctrl+r} to search further back.}
  \end{itemize}
\end{frame}

\begin{frame}{Linux: Keyboard shortcuts: alias}
  \begin{itemize}
    \item{Use \command{alias} to create your own abbreviations for long commands.}
    \item{Example: \command{alias s=`cd /share/ucapwhi/almanac\_project/'}}
    \item{It's boring to retype all your aliases at the start of your session. So instead put them in a batch file that autoexecutes at login - this will be named \filename{\textasciitilde/.bashrc} or something similar. }
  \end{itemize}
\end{frame}

\begin{frame}{Linux: Environment variables (1)}
  \begin{itemize}
    \item The operating system maintains a global namespace of `environment variables' to store configuration information.
    \item Use  \command{set} to see all environment variables;
    \item Use \command{echo \$<variable\_name>} to see the value of one environment variable (e.g. \command{echo \$PATH});
    \item Use \command{export \$FOO=`my\_string'} to set an environment variable \texttt{FOO}.
  \end{itemize}
\end{frame}

\begin{frame}{Linux: Environment variables (2)}
  \begin{itemize}
    \item Variables \texttt{PATH} and \texttt{PYTHONPATH} are used frequently (to maintain lists of directories in which to search for executable programs and Python modules, respectively).
    \item Linux has no equivalent of the Windows Registry; configuration is done via the directory structure and the environment variables.
  \end{itemize}
\end{frame}


\begin{frame}{Linux: Structure of commands}
  \begin{block}{Structure}
    \command{[command] -[option(s)] [argument]}
  \end{block}
  \begin{Examples}
     \command{ls -la} \\
     \command{mkdir my\_experiments} \\
     \command{cp hello.cpp new\_hello.cpp}
  \end{Examples}
\end{frame}


\begin{frame}{Linux: command reference}
  \begin{itemize}
  \item Use \command{man <command>} for short form info and \command{info <command>} for long form info on commands.
  \item \url{http://www.computerhope.com/unix.htm} is a useful reference for Linux commands.
\end{itemize}
\end{frame}

\begin{frame}{Linux: file management}
  \begin{itemize}
  \item Use \command{mkdir <dir>} to make a new directory
  \item Use \command{rm -rf <dir>} to delete a directory and its contents (including subdirectories). This is irreversible!
  \item Use \command{cp <source> <destination>} to copy a file and \command{mv <source> <destination>} to move a file.
  \item Use \command{scp <source> <destination>} to copy a file between servers. The syntax for the remote server is \filename{<username>@<servername>:<filename>}.
  \end{itemize}
\end{frame}

\begin{frame}{Linux: file contents}
  \begin{itemize}
  \item Use \command{cat <file>} to show the contents of a file as text and \command{xxd <file>} to show the contents of a file as bytes.
  \item Use \command{head <file>} or \command{tail <file>} to show the first or last few lines in a file - helpful if the file is large!
  \item Use \command{grep} to search for text with a file or files.
  \end{itemize}
\end{frame}

\begin{frame}{Linux: controlling processes}
  \begin{itemize}
  \item Use \command{top} to see all the processes running on a server (not just your own); type \command{q} to exit \command{top}. Helpful if someone seems to be hogging the processor!
  \item Use \command{kill} to stop one of your own processes.
  \item Use \command{watch} to run a command repeatedly.
  \item Use \command{nohup} or \command{screen} to let a command continue to run even after you exit your session.
  \item Use \command{\&} at the end of a command to have it run in the background; control is then immediately returned to you. Use \command{jobs} to see what is running in the background, and \command{fg} to move a job from background to foreground.
  \end{itemize}
\end{frame}

\begin{frame}{Linux: long command lines}
  \begin{itemize}
  \item Concatenate several commands onto one long command using semicolon e.g. \command{cd \textasciitilde;cat .bashrc;cd -}.
  \item Conversely use \command{\symbol{92}} to split one long command over two lines.
  \end{itemize}
\end{frame}

\begin{frame}{Linux: wildcards}
  \begin{itemize}
  \item Some commands take \textit{sets} of filenames as an argument. For example \command{rm <set of filenames>} will remove multiple files.
  \item Such sets can be given as a space-separated list (example \command{rm foo1.txt foo2.txt}), or else using \textit{wildcards} (example \command{rm foo*.txt}).
  \item The wildcards are \\ \command{*} (matches zero or more characters) and \\ \command{?} (matches preceisely one character).
  \end{itemize}
\end{frame}

\begin{frame}{Linux: vi}
  \begin{itemize}
  \item \command{vi} (or an improved version \command{vim}) is a text editor that is found on almost every UNIX machine.
  \item Some familiarity with \command{vi} is therefore useful, as it's often the fastest way to make small edits to text files.
  \item Bad news: the key commands for \command{vi} are \textit{very} different from those that have become standard on personal computers. Good news:  \command{vi} can be useful even if you know only a few commands.
  \item The minimum you need to know is how to exit \command{vi} if you get into it by accident: \command{esc : q! enter}.
  \end{itemize}
\end{frame}

\begin{frame}{Linux: redirection}
  \begin{itemize}
  \item Use \command{>} to \textit{redirect} the text output from a program to go to a file instead of the screen. Example: \command{ls >  listing.txt}.
  \item Converely use \command{<} to allow text input to a program to be taken from a file rather than the keyboard.
  \end{itemize}
\end{frame}



%\begin{frame}{Linux: Basic commands 1}
%  \begin{columns}
%    \column{.5\textwidth}
%    \begin{block}{navigation and help}
%      \texttt{pwd} \\
%      \texttt{ls -la} \\
%      \texttt{cd dir\_name} \\
%      \texttt{man command\_name} \\
%      \texttt{info command\_name} \\
%      \texttt{exit}
%    \end{block}
%    \begin{block}{copy or move}
%      \texttt{cp src dest} \\
%      \texttt{mv src dest} \\
%      \texttt{scp usr@host:file dest}
%    \end{block}
%
%    \column{.5\textwidth}
%    \begin{block}{create or delete}
%      \texttt{touch file.txt} \\ 
%      \texttt{mkdir dir\_name} \\ 
%      \texttt{rm -i file.txt}
%    \end{block}
%    \begin{block}{find and system info}
%      \texttt{whereis file} \\
%      \texttt{which} \\
%      \texttt{echo \$VAR\_NAME}
%    \end{block}
%    \begin{block}{file contents}
%      \texttt{cat file} \\
%      \texttt{more file} \\
%      \texttt{head file}
%    \end{block}    
%  \end{columns}
%\end{frame}
%
%
%\begin{frame}{Linux: Basic commands 2}
%  \begin{columns}
%    \column{.5\textwidth}
%    \begin{block}{special characters}
%      \texttt{\& (background)} \\
%      \texttt{; (combine)} \\
%      \texttt{\textbackslash$\,$ (next line)} \\
%      \texttt{* (wildcard)} \\
%      \texttt{| (pipe)} \\
%      \texttt{> (output)} \\            
%      \texttt{< (input)}
%    \end{block}
%    \begin{block}{text editors}
%      \texttt{emacs} \\
%      \texttt{vi} \\
%      \texttt{gedit}
%    \end{block}
%
%    \column{.5\textwidth}
%    \begin{block}{process management}
%      \texttt{kill}, \texttt{top}, \texttt{nohup}
%    \end{block}    
%    \begin{block}{compressed files}
%      \texttt{gunzip}, \texttt{tar} 
%    \end{block}
%    \begin{block}{images}
%      \texttt{gthumb} \\
%      \texttt{ds9} \\
%      \texttt{evince} \\
%      \texttt{eog}
%    \end{block}    
%    \begin{block}{publishing}
%      \texttt{latex}, \texttt{bibtex} 
%    \end{block}
%  \end{columns}
%\end{frame}
%

%  \begin{enumerate}
%    \item Go to your home directory and create a directory called \texttt{level\_0}.
%    \item Change directory to \texttt{level\_0}.
%    \item Find the name of the present working directory.
%    \item Make a directory \texttt{level\_1}, and move to it.
%    \item Create a file called \texttt{foo.txt} with contents ``This file contains the word bar''.
%    \item Add another line in \texttt{foo.txt} with contents ``This is the second line''.
%    \item Print the contents of \texttt{foo.txt} to the screen.
%    \item Search for the word \emph{bar} in \texttt{foo.txt}.
%    \item Go up one level, then remove the directory \texttt{level\_1} (and its contents).
%    \item Find the location of your python installation.
%    \item Find the values of the environment variables \texttt{PATH} and \texttt{LD\_LIBRARY\_PATH}.
%    \item Set the environment variable \texttt{MY\_VAR} to equal the absolute path to \texttt{level\_0}, and test that it has worked OK.
%    \item Add (i.e append) to the \texttt{PATH} the absolute path to \texttt{level\_0}, and test that it has worked OK.
%\begin{frame}[allowframebreaks,t]{Linux: Exercises}
%    \item Use the \texttt{man} command to find the option of \texttt{ls} that shows file sizes in human readable format.
%    \item Find the hostname, processor type and operating system version and write this info into a text file called \texttt{info.txt}.
%    \item List the people who are currently logged into the system.
%    \item Find which process is using the most CPU at the moment.
%    \item Find the IDs of the processes that you are running.
%  \end{enumerate}
%\end{frame}


\end{document}
