% http://www.ctan.org/tex-archive/macros/latex/contrib/beamer/examples
% http://latex.artikel-namsu.de/english/beamer-examples.html


\documentclass[dvipsnames]{beamer}
\usepackage{amsmath}
\usepackage{amssymb}
\usepackage{bm}
\usepackage{fancybox, graphicx}
\usepackage{listings}
\usepackage{color}
\usepackage{textcomp} % See https://tex.stackexchange.com/questions/145416/how-to-have-straight-single-quotes-in-lstlistings


\lstset{language=Python,upquote=true} % Format listings as appropriate for Python. Inexplicably we get problems if the language is set as part of the \begin{lstlisting} command.

% https://tex.stackexchange.com/questions/36030/how-to-make-a-single-word-look-as-some-code
\definecolor{light-gray}{gray}{0.95}
%\definecolor{light-green}{RGB}{0,250,0}
\definecolor{light-green}{cmyk}{0.07,0,0.07,0}
\newcommand{\command}[1]{\colorbox{light-gray}{\texttt{#1}}}
\newcommand{\filename}[1]{\colorbox{light-green}{\texttt{#1}}}
\newcommand{\server}[1]{\textcolor{BrickRed}{\texttt{#1}}}


\usetheme{boxes}
\usecolortheme{beaver}


\title{Introduction to the UNIX Command Line}
\author{Lorne~Whiteway \\ lorne.whiteway.13@ucl.ac.uk}
\institute{Astrophysics Group\\  Department of Physics and Astronomy\\  University College London}
\date{30 January 2025}

\begin{document}

\frame{\titlepage}

\begin{frame}{Where to find this presentation}
    Find the presentation at \alert{\url{https://tinyurl.com/ytt3kdm3}}.\\
\end{frame}


\begin{frame}{Goals of presentation}
  \begin{itemize}
    \item How to access UCL UNIX systems
    \item How to use the UNIX command line
  \end{itemize}
\end{frame}


\begin{frame}{Information on the Web}
  \begin{block}{Astrophysics Wiki}
    \url{https://liveuclac.sharepoint.com/sites/PhysAstAstPhysGrp}
    This Wiki is freely viewable and editable by all members of the department. Please use it to record information that you think will be useful to others (including your future self). Be bold!
  \end{block}
  \begin{block}{UCL Research Computing Platforms}
    \url{https://www.rc.ucl.ac.uk/}
  \end{block}
  \begin{block}{Stack Overflow}
    \url{http://stackoverflow.com/}  \ (But will it survive ChatGPT?)
  \end{block}
\end{frame}


\begin{frame}{Computing Environment for Astrophysics}
  \begin{itemize}
  \item Large datasets requiring substantial processing followed by sophisticated statistical analysis
  \item Calculations often done on specialised \textit{high-performance computing} (HPC) machines having large filesystems and large RAM; calculations are often broken into pieces that can be run simultaneously (`in parallel') across many processors.
  \item Much useful software is made freely available within the community. Software quality is usually high; documentation quality is more variable.
  \item Many users write their own software.
  \end{itemize}
\end{frame}


\begin{frame}{Local Computing Environment}
  \begin{block}{You will have your own local machine, which might be:}
    \begin{itemize}
      \item PC (Windows)
      \item Mac
      \item Linux
    \end{itemize}
  \end{block}
  \begin{block}{Also there are shared Linux machines:}
    \begin{itemize}
      \item General purpose Astrophysics servers available from outside UCL: \server{zuserver1} and \server{zuserver2}
      \item UCL Cosmology HPC clusters: \server{splinter} and \server{hypatia}
      \item Other UCL clusters: \server{Myriad} and \server{Kathleen}
      \item National clusters: \server{DiRAC} (UK) and \server{NERSC} (US)
    \end{itemize}
  \end{block}
\end{frame}


\begin{frame}{Work patterns}
  \begin{block}{Several work patterns are possible:}
    \begin{itemize}
      \item Write and test a program on your local machine; use the local machine to remotely connect to a server; upload the program to the server and run it there;
      \item Or do all your work locally (requires small data sets);
      \item Or use the local machine to remotely connect to a server and do all your work there.
    \end{itemize}
  \end{block}
\end{frame}


\begin{frame}{Accessing remote machines}
  \begin{block}{Credentials}
    \begin{itemize}
      \item You will need a \textit{username} and \textit{password} for any machine that you want to access.
      \item Contact Edd Edmondson (e.edmondson@ucl.ac.uk) or John Deacon (j.deacon@ucl.ac.uk) to get these credentials.
    \end{itemize}
  \end{block}
  \begin{block}{The full names of the Astro servers are:}
    \begin{itemize}
      \item \server{zuserver1.star.ucl.ac.uk}
      \item \server{zuserver2.star.ucl.ac.uk}
      \item \server{splinter-login.star.ucl.ac.uk}
      \item \server{hypatia-login.hpc.phys.ucl.ac.uk}
    \end{itemize}
  \end{block}
\end{frame}


\begin{frame}{Software for connecting}
  \begin{block}{How to connect to a shared machine}
    \begin{itemize}
      \item Windows PC: use \command{PuTTY} (or \command{MobaXterm}, which uses \command{PuTTY});
      \item Mac: go to the Terminal window and use \command{ssh};
      \item Linux machine: go to the Terminal window and use \command{ssh}.
    \end{itemize}
  \end{block}
\end{frame}


\begin{frame}{Visibility}
  \begin{block}{}
    \begin{itemize}
    \item \server{zuserver*} ($\equiv$ \server{zuserver1} or \server{zuserver2}) can be seen from anywhere.
    \item If you are on the UCL network then you can see any Astro or UCL HPC server.
    \item If you are not on the UCL network then to connect to Astro or UCL HPC servers (except \server{zuserver*}) you must set up a \textit{Virtual Private Environment}. For details see \url{https://www.ucl.ac.uk/isd/services/get-connected/ucl-virtual-private-network-vpn}. An alternative is to logon to \server{zuserver*} and from there logon to the server.
    \end{itemize}
  \end{block}
\end{frame}


\begin{frame}{Using PuTTY for remote connections from Windows}
  \begin{itemize}
    \item If you don't have PuTTY you can download it from \url{http://www.putty.org/}.
    \item On the `Connection/SSH/X11' tab, click on `enable X11 forwarding' and set `X display location' to `localhost:0' - this is necessary for handling graphical output.
    \item On the Session tab, set the Host Name as appropriate e.g. \server{zuserver1.star.ucl.ac.uk}.
  \end{itemize}
\end{frame}


\begin{frame}{Using ssh for remote connections from Mac and Linux}
  \begin{itemize}
    \item Syntax: \command{ssh -YC username@servername}
    \item The `Y' option is necessary for handling graphical output.
  \end{itemize}
\end{frame}


\begin{frame}{X-Windows client}
  \begin{itemize}
    \item If the remote program that you are running produces graphical output, then you must have a program (an `X-Windows client') running on your local machine to display this graphical output.
    \item On Windows you can use XMing (\url{https://sourceforge.net/projects/xming/}) or Exceed (available on the UCL Desktop).
    \item On Mac you can use XQuartz.
    \item On Linux you don't need to do anything special - the graphical interface is already an X-server.
  \end{itemize}
\end{frame}


\begin{frame}{Unix}
  \begin{itemize}
    \item Unix and Unix-like computer operating systems have become the industry standard for scientific research.
    \item Unix was started in 1969 by Ken Thompson, Dennis Ritchie, and others.
    \item Unix is multi-tasking and multi-user, with a modular design: the operating system provides various single-purpose tools that may be linked together for more complicated tasks.
    \item Unix has a reputation for efficiency, robustness, and security.
   \end{itemize}
\end{frame}


\begin{frame}{Linux}
  \begin{itemize}
    \item Linux is a popular `Unix-like' operating system.
    \item Linux was created by  Linus Torvalds, and was first released in 1991. His motivation was to avoid restrictive software licenses.
    \item \server{splinter} and \server{hypatia} use a version of Linux called `CentOS'.
    \item Free and open source.
   \end{itemize}
\end{frame}


\begin{frame}{Command shell}
  \begin{itemize}
    \item In Linux you will use a `command shell'; a text environment in which you type commands and receive text output.
    \item Not GUI! Reflects the hardware limitations current when Unix was created. Low-tech and reliable e.g. for remote access.
    \item Various command shell programs are used: \command{bash}, \command{csh}, \command{tcsh}, \command{zsh}, etc. Use \command{echo \$0} to see which one is in use.
    \item This presentation assumes \command{bash}. Other shells may use different names for some of the commands discussed.
    \item New accounts on UCL Astro servers get given \command{tcsh} as a shell. But if you don't use \texttt{Starlink} software, it's safe to ask to be switched to \command{bash}.
   \end{itemize}
\end{frame}


\begin{frame}{Directory structure}
  \begin{itemize}
    \item Everything is organised around files (which may be data files or program files i.e. instructions to be executed).
    \item Files live in directories. There is a hierarchical tree structure of directories.
    \item The \textit{root} directory (the base of the directory tree) is called \filename{/}.
    \item Sample file name: \filename{/home/ucapwhi/foo.txt}
    \item Note use of slash `\texttt{/}', not backslash `\texttt{\symbol{92}}' as in Windows.
    \item Case sensitivity: `Foo' and `foo' are different strings.
  \end{itemize}
\end{frame}


\begin{frame}{Working directory}
  \begin{itemize}
    \item The shell is always pointed at one particular directory, known as the \textit{working directory}.
    \item Use \command{pwd} (`print working directory') to see the current working directory.
    \item Refer to files in the working directory simply via the file name (example \filename{foo.txt}) and refer to files in other directories by directory name plus file name (example \filename{/home/ucapwhi/foo.txt}).
  \end{itemize}
\end{frame}


\begin{frame}{Abbreviations for directories}
  \begin{itemize}
    \item Full stop \filename{.} is an abbreviation for the working directory.
    \item Two full stops \filename{..} is an abbreviation for the parent of the working directory.
    \item Hyphen \filename{-} is an abbreviation for the most-recent previous working directory -- useful if you need to flip back and forth between directories!
    \item Tilde \filename{\textasciitilde} is an abbreviation of the user's \textit{home} directory e.g. \filename{/home/ucapwhi/}. Many configuration files are located here by default.
  \end{itemize}
\end{frame}


\begin{frame}{Navigating the directory tree}
  \begin{itemize}
    \item Use \command{cd} to change working directory. Example: \command{cd ../data/des/}.
    \item Use \command{ls} to list the files in the current working directory; \\use \command{ls <dir>} to list the files in another directory.
  \end{itemize}
\end{frame}


\begin{frame}{Keyboard speedups}
  \begin{itemize}
    \item Linux has several speedups that make it efficient to use the keyboard to type commands.
    \item Keyboard interfaces predate more modern graphical interfaces; they are less friendly for new users, but are low-tech, robust, and very efficient for experienced users.
  \end{itemize}
\end{frame}


\begin{frame}{Keyboard shortcuts: tab}
  \begin{itemize}
    \item Use the \command{tab} key to autocomplete commands, directory names and filenames.
    \item If what you have typed so far doesn't have a unique autocompletion, then it will complete up to the first ambiguous character.
    \item This will influence your naming conventions for directories and files!
  \end{itemize}
\end{frame}


\begin{frame}{Keyboard shortcuts: up and down arrows}
  \begin{itemize}
    \item Use \command{up arrow} to scroll backwards through previous commands, and then \command{down arrow} to scroll forwards again.
    \item Follow with \command{Enter} to execute an old command that you have scrolled back to, or \command{ctrl+c} to cancel.
  \end{itemize}
\end{frame}


\begin{frame}{Keyboard shortcuts: ctrl+r}
  \begin{itemize}
    \item Use \command{ctrl+r} to search backwards through previous commands (\textit{reverse-i-search}).
    \item Type a substring (not necessarily initial) of the sought-for command.
    \item Example: \command{ctrl+r push} will bring up the most recent command that included the substring \command{push}.
    \item Follow with \command{Enter} to execute, \command{ctrl+c} to cancel, or \command{ctrl+r} to search further back.
  \end{itemize}
\end{frame}


\begin{frame}{Keyboard shortcuts: alias}
  \begin{itemize}
    \item Use \command{alias} to create your own abbreviations for long commands.
    \item Example: \command{alias s=`cd /share/ucapwhi/almanac\_project/'}
    \item It's boring to retype all your aliases at the start of your session. So instead put them in a batch file that autoexecutes at login - this will be named \filename{\textasciitilde/.bashrc} or something similar.
  \end{itemize}
\end{frame}


\begin{frame}{Environment variables (1)}
  \begin{itemize}
    \item The operating system maintains a global namespace of `environment variables' to store configuration information.
    \item The following commands are specific to \command{bash} - other shells use different commands.
    \item Use  \command{set} to see all environment variables;
    \item Use \command{echo \$<variable\_name>} to see the value of one environment variable (e.g. \command{echo \$PATH});
    \item Use \command{export \$FOO=`my\_string'} to set an environment variable \texttt{FOO}.
  \end{itemize}
\end{frame}


\begin{frame}{Environment variables (2)}
  \begin{itemize}
    \item Variables \texttt{PATH} and \texttt{PYTHONPATH} are used frequently (to maintain lists of directories in which to search for executable programs and Python modules, respectively).
    \item Linux has no equivalent of the Windows Registry; configuration is done via the directory structure and the environment variables.
  \end{itemize}
\end{frame}


\begin{frame}{Structure of commands}
  \begin{block}{Structure}
    \command{[command] -[option(s)] [argument]}
  \end{block}
  \begin{Examples}
     \command{ls -la} \\
     \command{mkdir my\_experiments} \\
     \command{cp hello.cpp new\_hello.cpp}
  \end{Examples}
\end{frame}


\begin{frame}{Command reference}
  \begin{itemize}
  \item For help with \command{<command>} (in increasing order of verbosity): \\ \command{<command> -h} or \command{<command> --help} \\ \command{man <command>} \\ \command{info <command>}
  \item \url{http://www.computerhope.com/unix.htm} is a useful reference for Linux commands.
\end{itemize}
\end{frame}


\begin{frame}{File management}
  \begin{itemize}
  \item Use \command{mkdir <dir>} to make a new directory
  \item Use \command{rm -rf <dir>} to delete a directory and its contents (including subdirectories). This is irreversible!
  \item Use \command{cp <source> <destination>} to copy a file and \command{mv <source> <destination>} to move a file.
  \item Use \command{scp <source> <destination>} to copy a file between servers. The syntax for the remote server is \filename{<username>@<servername>:<filename>}. See also \command{rsync}.
  \end{itemize}
\end{frame}


\begin{frame}{File contents}
  \begin{itemize}
  \item Use \command{cat <file>} to show the contents of a file as text and \command{xxd <file>} to show the contents of a file as bytes.
  \item Use \command{head <file>} or \command{tail <file>} to show the first or last few lines in a file - helpful if the file is large!
  \item Use \command{grep} to search for text with a file or files.
  \end{itemize}
\end{frame}


\begin{frame}{Controlling processes}
  \begin{itemize}
  \item Use \command{top} to see all the processes running on a server (not just your own); type \command{q} to exit \command{top}. Helpful if someone seems to be hogging the processor!
  \item Use \command{kill} to stop one of your own processes.
  \item Use \command{watch} to run a command repeatedly.
  \item Use \command{nohup} or \command{screen} to let a command continue to run even after you exit your session.
  \item Use \command{\&} at the end of a command to have it run in the background; control is then immediately returned to you. Use \command{jobs} to see what is running in the background, and \command{fg} to move a job from background to foreground.
  \end{itemize}
\end{frame}


\begin{frame}{Long command lines}
  \begin{itemize}
  \item Concatenate several commands onto one long command using semicolon e.g. \command{cd \textasciitilde;cat .bashrc;cd -}.
  \item Conversely use \command{\symbol{92}} to split one long command over two lines.
  \end{itemize}
\end{frame}


\begin{frame}{Wildcards}
  \begin{itemize}
  \item Some commands take \textit{sets} of filenames as an argument. For example \command{rm <set of filenames>} will remove multiple files.
  \item Such sets can be given as a space-separated list (example \command{rm foo1.txt foo2.txt}), or else using \textit{wildcards} (example \command{rm foo*.txt}).
  \item The wildcards are \\ \command{*} (matches zero or more characters) and \\ \command{?} (matches preceisely one character).
  \end{itemize}
\end{frame}


\begin{frame}{vim}
  \begin{itemize}
  \item \command{vim} (an improved version of \command{vi}) is a text editor that is found on almost every UNIX machine.
  \item Some familiarity with \command{vim} is therefore useful, as it's often the fastest way to make small edits to text files.
  \item Bad news: the key commands for \command{vim} are \textit{very} different from those that have become standard on personal computers. Good news:  \command{vim} can be useful even if you know only a few commands.
  \item The minimum you need to know is how to exit \command{vim} if you get into it by accident: \command{esc : q! enter}.
  \end{itemize}
\end{frame}


\begin{frame}{Redirection}
  \begin{itemize}
  \item Use \command{>} to \textit{redirect} the text output from a program to go to a file instead of the screen. Example: \command{ls >  listing.txt}.
  \item Conversely use \command{<} to allow text input to a program to be taken from a file rather than the keyboard.
  \item Use the pipe symbol \command{|} to allow output from one program to be used an input for another. Example: \command{ifconfig | grep UP}.
  \end{itemize}
\end{frame}


\begin{frame}{bash scripts}
  \begin{itemize}
  \item bash comes with a Turing-complete scripting language in which you can write programs.
  \item Such scripts are usually used for control of processing (rather than for e.g. actual processing of data).
  \item The language has some idiosyncrasies (e.g. \command{fi} instead of \command{end\_if}).
  \item Every programs return an integer \textit{return code} ($0 = $ success) that can be used in conditional statements.
  \item It is common to use the extension \filename{.sh} for such files.
  \end{itemize}
\end{frame}


\begin{frame}{Example bash script}
  \includegraphics[scale=0.9]{bash\_script.jpg}
\end{frame}


\begin{frame}{How to run a bash script}
    \command{bash myscript.sh} \\
  \vspace{1cm}
  Alternatively make the script executable:
  \begin{itemize}
  \item Add \command{\#!/usr/bin/env bash} at the top of the file
   \item Then \command{chmod +x myscript.sh}
     \end{itemize}
     Then just need to call \command{./myscript.sh}
\end{frame}


\begin{frame}{Scripts and environment variables}
\begin{itemize}
\item It's natural to write a script that sets environment variables.
\item But this doesn't work! The script gets a \textit{copy} of the environment; this copy gets amended by the script, but is then discarded when the script stops.
\item So instead use \command{source} e.g. \command{source my\_set\_environ\_vars.sh}
\end{itemize}
\end{frame}


\begin{frame}{Modules}
\begin{itemize}
\item Both \server{splinter} and \server{hypatia} use the \command{module} package to give access to software packages (typically by setting appropriate environment variables).
\item Example: on \server{splinter}, \command{module load dev\_tools/oct2021/hdf5-1.12.1} makes version \texttt{1.12.1} of the \command{hdf5} package available.
\item Use \command{module avail} to see what you can load and \command{module list} to see what you have loaded.
\item If desired, you can extend this system to include your own software.
\end{itemize}
\end{frame}


\begin{frame}{HPC environments}
\begin{itemize}
\item \server{splinter} and \server{hypatia} are \textit{high-performance computing} (HPC) clusters.
\item Each has many linked computers (called \textit{nodes}). Most nodes are \textit{compute} nodes; these are available for processing. The \textit{login} node (which you login to) is special. Use it for control, for editing and for compiling, but not for heavy processing!
\item \command{srun --pty bash} will give you a session on one of the compute nodes.
\item Alternatively you can use \command{sbatch} to launch an unattended job on one or several compute nodes.
\end{itemize}
\end{frame}


\end{document}
