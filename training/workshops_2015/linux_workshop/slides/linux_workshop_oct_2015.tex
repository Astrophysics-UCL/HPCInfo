% http://www.ctan.org/tex-archive/macros/latex/contrib/beamer/examples
% http://latex.artikel-namsu.de/english/beamer-examples.html

\documentclass{beamer}
\usepackage{amsmath}
\usepackage{amssymb}
\usepackage{bm}
\usepackage{fancybox, graphicx}
\usepackage{listings}


\lstdefinestyle{customc}{
  belowcaptionskip=1\baselineskip,
  breaklines=true,
  %frame=L,
  xleftmargin=\parindent,
  language=bash,
  showstringspaces=false,
  basicstyle=\footnotesize\ttfamily,
  keywordstyle=\bfseries\color{green!40!black},
  commentstyle=\itshape\color{purple!40!black},
  identifierstyle=\color{blue},
  stringstyle=\color{orange},
  deletekeywords={login}, % Needed to prevent spurious colouring of `login' where it appears in `splinter-login'.
}

\lstset{style=customc}

\usetheme{Dresden}


\title[Linux  Workshop] % (optional, use only with long paper titles)
{Linux  Workshop}

\author[Balan, Whiteway, Wilson, Palmese] % (optional, use only with lots of authors)
{S.~T.~Balan, L.~Whiteway, T.~Wilson, A.~Palmese }

\institute[UCL]
{
  Department of Physics and Astronomy\\
  University College London
}
\date[Linux  2015]
{15 October 2015}

\subject{Linux }

\begin{document}

\frame{\titlepage}

\section{Basics}

\begin{frame}{Where to find this presentation}
  \begin{block}{url}
    \url{https://github.com/Astrophysics-UCL/HPCInfo/blob/master/training/workshops_2015/linux_workshop/slides/linux_workshop_oct_2015.pdf}
  \end{block}
\end{frame}


\begin{frame}{What will you learn?}
  \begin{itemize}
    \item{In this talk:}
    \begin{itemize}
      \item Accessing Astrophysics group machines
      \item Using the Linux console for your research
    \end{itemize}
    \item{In the next talk:}
    \begin{itemize}
      \item How to run programs on High Performance Computing (HPC) machines
    \end{itemize}
  \end{itemize}
\end{frame}

\begin{frame}{Information on the Web}
  \begin{block}{Astrophysics Wiki}
    \url{https://wiki.ucl.ac.uk/display/PhysAstAstPhysGrp/Main+Page}
    This Wiki is freely viewable and editable by all members of the department. Please use it to record information that you think will be useful to others (including your future self). Be bold!
  \end{block}

  \begin{block}{UCL Research Computing Platforms}
    \url{https://wiki.rc.ucl.ac.uk/wiki/Main_Page}
  \end{block}
  
  \begin{block}{Stack Overflow}
    \url{http://stackoverflow.com/}
  \end{block}
  
\end{frame}


\begin{frame}{Command shell}
  \begin{itemize}
    \item{You will be using a `command shell'.}
    \item{This is a text-based environment in which you type commands and text output.}
    \item{Not GUI! Reflects the hardware limitations current when Unix was created. Low-tech and reliable e.g. for remote access.}
    \item{Various command shell programs in use: \textit{bash}, \textit{csh}, \textit{tcsh},...}
   \end{itemize}
\end{frame}

\begin{frame}[fragile]{Accessing machines remotely}
  \alert{You will need a \emph{username} and \emph{password}}
  \begin{block}{}
    \lstinputlisting{login.sh}
  \end{block}
\end{frame}

\begin{frame}{Directory structure}
  \begin{itemize}
    \item{Everything is organised around files (which may be data files or program files i.e. instructions to be executed).}
    \item{Files live in directories. There is a hierarchical tree structure of directories.}
    \item{Sample file name: /share/splinter/ucapwhi/des/foo.txt}
    \item{Note use of slash `/', not backslash `\textbackslash' as in Windows.}
    \item{Case sensitivity: `Foo' and `foo' are different strings.}
  \end{itemize}
\end{frame}

\begin{frame}{Special symbols for directories}
  \begin{table}[ht]
    \centering
    \begin{tabular}{c l}
      \\ [-2ex]
      Symbol & Meaning \\ [.5ex]
      \hline \\ [-2ex]
      / & Top of the directory tree (the root directory) \\
      . & Current directory \\
      .. & Parent of the current directory \\
      \textasciitilde & User's `home' directory
    \end{tabular}
  \end{table}
\end{frame}

\begin{frame}[fragile]{Structure of commands}
  \begin{block}{Structure}
    \lstinputlisting{command.sh}
  \end{block}
  \begin{example}
    \lstinputlisting{command_example.sh}
  \end{example}
\end{frame}


\begin{frame}[fragile]{Linux console cheat sheat I}
  \fontsize{7pt}{7}\selectfont
  \begin{columns}
    \column{.5\textwidth}
    \begin{block}{navigation and help}
      \lstinputlisting{navigation.sh}
    \end{block}
    \begin{block}{copy or move}
      \lstinputlisting{copy_move.sh}
    \end{block}

    \column{.5\textwidth}
    \begin{block}{create or delete}
      \lstinputlisting{make_delete.sh}
    \end{block}
    \begin{block}{find or search}
      \lstinputlisting{find_search.sh}
    \end{block}
  \end{columns}
\end{frame}

\begin{frame}[fragile]{Linux console cheat sheat II}
  \fontsize{7pt}{7}\selectfont
  \begin{columns}
    \column{.5\textwidth}
    \begin{block}{file contents}
      \lstinputlisting{file_contents.sh}
    \end{block}
    \begin{block}{process management}
      \lstinputlisting{process_management.sh}
    \end{block}

    \column{.5\textwidth}
    \begin{block}{ssh}
      \lstinputlisting[language=bash]{ssh_scp.sh}
    \end{block}
    \begin{block}{system info}
      \lstinputlisting[language=bash]{system_info.sh}
    \end{block}
  \end{columns}
\end{frame}


\begin{frame}[fragile]{Linux console cheat sheat III}
  \fontsize{7pt}{7}\selectfont
  \begin{columns}
    \column{.5\textwidth}
    \begin{block}{\& ; $|$ > <}
      \lstinputlisting[language=bash]{wild_card.sh}
    \end{block}
    \begin{block}{Text editors}
      \lstinputlisting[language=bash]{text_editors.sh}
    \end{block}

    \column{.5\textwidth}
    \begin{block}{web}
      \lstinputlisting[language=bash]{web.sh}
    \end{block}
    \begin{block}{publishing}
      \lstinputlisting[language=bash]{publishing.sh}
    \end{block}
  \end{columns}
\end{frame}

\begin{frame}[fragile]{Linux console cheat sheat IV}
  \fontsize{7pt}{7}\selectfont
  \begin{columns}
    \column{.5\textwidth}
    \begin{block}{compressed files}
      \lstinputlisting[language=bash]{compress.sh}
    \end{block}
    \begin{block}{images}
      \lstinputlisting[language=bash]{images.sh}
    \end{block}

    \column{.5\textwidth}
    \begin{block}{development}
      \lstinputlisting{development.sh}
    \end{block}
    \begin{block}{scientific}
      \lstinputlisting{scientific.sh}
    \end{block}
  \end{columns}
\end{frame}

\begin{frame}{Exercises I}
  \fontsize{8pt}{8}\selectfont
  \begin{enumerate}
    \item Go to your home directory and create a directory called \texttt{linux\_hpc\_workshop}.
    \item Change directory to \texttt{linux\_hpc\_workshop}.
    \item What is the present working directory?
    \item Make a directory \texttt{level\_1/level\_2}, and move to \texttt{level\_1/level\_2} in one command.
    \item Move back to the previous directory.
    \item Remove the directory \texttt{level\_1} (and its contents).
    \item Make a symbolic link to \texttt{usr/lib} in the current directory called \texttt{my\_sybolic\_link}.
    \item Create a file called \texttt{foo.txt} with contents ``This file contains the word foo''.
    \item Add another line in \texttt{foo.txt} called ``This is the second line''.
    \item Check if it worked.
    \item Search for the phrase \emph{foo} in \texttt{foo.txt}.
  \end{enumerate}
\end{frame}


\begin{frame}{Exercises II}
  \fontsize{8pt}{8}\selectfont
  \begin{enumerate}
    \item Find the location of your python installation.
    \item Find the installation location(s) of \texttt{liblapack.a}.
    \item Find whether an object \texttt{daxpy} is in \texttt{liblapack.a}.
    \item Find the value the environment variable \texttt{PATH} and \texttt{LD\_LIBRARY\_PATH}.
    \item Set the environment variable \texttt{MY\_LINUX\_HPC\_VAR} to the absolute path to \texttt{linux\_hpc\_workshop}.
    \item Add (i.e append) to the \texttt{PATH} the absolute path to \texttt{linux\_hpc\_workshop}.
    \item Use the \textit{source} command do the last two steps from source file.
    \item Use the \textit{man} command to find the option of \texttt{ls} that shows the output in Kilobyte,Megabyte.
  \end{enumerate}
\end{frame}

\begin{frame}{Exercises III}
  \fontsize{8pt}{8}\selectfont
  \begin{enumerate}
    \item Find hostname, processor type and operating system version and write this info into a text file called \texttt{info.txt}.
    \item List the people who are currently logged into the system.
    \item Find the process that is taking most of the CPU at the moment.
    \item Find the IDs of the processes that you are running.
    \item Make a directory called \texttt{to\_be\_compressed}. Add the files \texttt{hello.cpp} and \texttt{hello.py} in this dir.
    Now compress this directory using \textit{tar} and \textit{zip}.
    \item Delete the dirctory \texttt{to\_be\_compressed} and extract the files from \texttt{to\_be\_compressed.tar.gz}.
    \item Use \textit{wget} to download  files from \url{ftp://heasarc.gsfc.nasa.gov/software/fitsio/c/cfitsio3370.tar.gz}.
    \item What is the size of the item you just downloaded in MB?
    \item Find the number of occurrences of the phrase \texttt{table} in all the files with extension \texttt{.h}.
    \item Remove all the files with extension \texttt{.h}.
    \item Copy the files with extension \texttt{.c} into a new directory \texttt{c\_files}.
  \end{enumerate}
\end{frame}


\end{document}
